% [page size, font size, recto-verso]{style of the document}
\documentclass[a4paper, 12pt, oneside]{report}

% Encoding of this file. On Linux, all files are UTF8 encoded.
\usepackage[utf8x]{inputenc}

% To write mathematical formula:
% \begin{align} ... \end{align}, \begin{equation} ... \end{equation}
\usepackage{amsmath, amssymb}

% To insert graphics:
% \includegraphics[options]{path}
\usepackage{graphicx}

% The allowed type of image files.
\DeclareGraphicsExtensions{.pdf, .png}

% To include code file without any layout
% \verbatiminput{path}
\usepackage{verbatim}

% To include PDF files
% \includepdf[pages = z,x-y]{path}
\usepackage{pdfpages}

% To do tables on many pages
% \begin{longtable}{define the columns} ... \end{longtable}
\usepackage{longtable}


\textwidth=16cm
\oddsidemargin=0pt
\evensidemargin=0pt

%% The 3 variables which you must initialize:
% Title of the document,
\newcommand{\titleVariable}{Remote control using a keyboard}
% The author(s),
\newcommand{\authorVariable}{Marko Doornbos}
% The date of the first release.
\newcommand{\firstRelease}{Nov 26, 2012}

\usepackage{hyperref}
% To create all hypertext links for the bibliography, the figures, the bookmarks etc.
% \url{http://www.thewebsite.org}
\hypersetup{
%   backref=true,
%   pagebackref=true,
%   hyperindex=true,
    colorlinks=true,
    breaklinks=true,
    urlcolor= blue,
    linkcolor= blue,
%   bookmarks=true,
    bookmarksopen=true,
    pdftitle={\titleVariable},
    pdfauthor={\authorVariable},
    pdfsubject={Documentation}
}

\title{\titleVariable}
\author{\authorVariable}


\date{\centering First release: \firstRelease \\ Last modification: \today}

\begin{document}
\maketitle

\chapter{the remote module}

This explains how to use the $remote$ module. 
The module is used to read keyboard presses from a separate window on the executing computer.

\section{Introduction}

The $remote$ module can be found in the $src/vision$ folder. It reads keyboard key presses and writes them to the memory. 

Keep in mind that key presses are only read if the window generated for this purpose is in focus.

\section{How to start}

To have the module running in your behavior, add $remote1 = remote$ to the $modules\_settings:$ part and set it to run on the host you want, probably localhost. 

So like: $localhost = remote1$ in the $modules:$ section of the config file.

\section{How to read from memory}

Entered commands are stored in memory using the $"pioneer\_command"$ key. Use that to retrieve them. 

They are stored as tuple of the time the command was stored and the dict containing the actual data. This is the default format for items in memory.

The following commands can be stored under this key by pressing the corresponding keys on the keyboard: 

\begin{itemize}
\item{}'w' is "forward"
\item{}'a' is "left"
\item{}'d' is "right"
\item{}'s' is "backward"
\item{}'q' is "stop"
\item{}'r' is "terminate"
\end{itemize}

While these commands are meant to instruct movement, they could be used for other purposes as well. However, it is preferred if that is not done.

\chapter{The remotecontrol behavior}

\section{General information}

This behavior uses the key presses read by the $remote$ module to make the pioneer move.

When making a config to run this behavior, make sure you have a Pioneer and have the $remote$ module running, as described above.
While running this behavior, the robot will move according to the keypresses as explained above.

Keep in mind that this behavior does not use obstacle avoidance, so BE CAREFUL WHEN USING IT!

\section{Notes}

This behavior stops when you press the 'r' key.

\end{document}

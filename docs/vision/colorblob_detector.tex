\documentclass[a4paper, 10pt]{article}

\usepackage{fullpage}
\usepackage[english]{babel}
\usepackage{graphicx}
\usepackage{multirow}
\usepackage{array}
\usepackage{listings}
\usepackage{url}
\usepackage{hyperref}
\usepackage{courier}

\hypersetup{
	colorlinks=true,
	breaklinks=true,
	urlcolor= black,
	linkcolor= black,
    citecolor= black,
}

\lstset{language=c, frame=lines, basicstyle=\tt, tabsize=4, numberstyle=\tt}


\title{HOWTO - Start with the ColorBlob Detector \\ \small{Related courses: Robotica, PAS1 and PAS2}}
\author{Team BORG}

\begin{document}
\maketitle

\tableofcontents

\section{Introduction}

This document introduces you to the ColorBlob detector module.

\textbf{Please note that this document is a work in progress and may be updated in time.}

\section{Installation}

\begin{enumerate}
    \item Install python, opencv and pygame on a linux box:
\begin{lstlisting}
# sudo apt-get update
# sudo apt-get install python python-opencv python-pygame
\end{lstlisting}
    \item Use Python \lstinline{pip} or install manually if the above does not work.
\end{enumerate}

\section{Basic Training Procedure}

The basic training procedure is as follows.

\begin{enumerate}
    \item Create an empty directory.
    \item Create as many \lstinline{*.colorblob} files as objects (color blobs) you like to detect inside this directory. 
        Make sure the files are \textbf{empty}.
    \item Add the blobdetector module to your config file and specify the absolute path of the directory (see \lstinline{config_blob} as a example).
    \item Make sure that the ``train'' configuration parameter inside the config file is set to ``True''.
    \item Now execute the start.sh script or the startGUI.sh application (and thus the blobdetection module as well). 
        The videofeed of the Nao (or any other video input that you have selected in the config file) should now be visible.
    \item Use the ``o'' and ``p'' keys to select the \lstinline{*.colorblob} file.
    \item Now click on the object that you like to be detected. 
        Make sure to click a few times while placing the object in different angles and locations.
        (The more you click the more processing power it requires to detect the object.)
    \item Do this for each object and press 'k' to save all *.colorblob files.
\end{enumerate}

\section{Basic Usage}

Basic usage in your behavior is as follows:
\begin{enumerate}
    \item Use the same name as the \lstinline{*.colorblob} file to retreive the observation from memory.
    \item The observation has 3 fields: ``x'', ``y'' and ``size''. 
        The ``x'' and ``y'' is the absolute location of the object and ``size'' is the relative size of the object (a number between 0 and 1).
    \item See the ``exampleblobdetector'' behavior and config file for more details.
\end{enumerate}

\section{Keyboard Shortcuts}

The following keyboard shortcuts are available during the training procedure (e.d. the train parameter is set to true in the config file).

\begin{enumerate}
    \item[\textbf{f}:] To freeze the current image.
    \item[\textbf{l}:] To load the (specified) color cube list.
    \item[\textbf{s}:] To save the (specified) color cube list.
    \item[\textbf{d}:] To empty the (specified) color cube list.
    \item[\textbf{a}:] To increase the cube size for the current blob.
    \item[\textbf{z}:] To decrease the cube size for the current blob.
    \item[\textbf{u}:] To remove the last newly added color from the color cube list.
    \item[\textbf{p}:] To select the next \lstinline{*.colorblob} file.
    \item[\textbf{o}:] To select the previous \lstinline{*.colorblob} file.
    \item[\textbf{k}:] To save all colorblob files.
\end{enumerate}

\section{Training From a Behavior}

To train the colorblob module on the fly using a behavior, you are required to send a specific command to the colorblob module using the visioncontroller.
You can do this using the \lstinline{send_command} method of the visioncontroller. 

The command to send is ``train'' and the parameters to use are as follows:
\begin{enumerate}
    \item \lstinline{box}: Specifies the top-left corner (tuple) and bottom-right corner (tuple) of the square to sample.
    \item \lstinline{sample_count}: Specifies the amount of pixels to sample.
    \item \lstinline{name}: Specifies the name of the file as specified in the \lstinline{*.colorblob} file directory.
\end{enumerate}

See for more information the ``\lstinline{exampleblobdetectorface}'' behavior and the \\
``\lstinline{config_example_blobdetector_face}'' configuration file.

\end{document}


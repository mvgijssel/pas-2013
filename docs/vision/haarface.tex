\documentclass[a4paper, 10pt]{article}

\usepackage{fullpage}
\usepackage[english]{babel}
\usepackage{graphicx}
\usepackage{multirow}
\usepackage{array}
\usepackage{listings}
\usepackage{url}
\usepackage{hyperref}
\usepackage{courier}

\hypersetup{
	colorlinks=true,
	breaklinks=true,
	urlcolor= black,
	linkcolor= black,
    citecolor= black,
}

\lstset{language=c, frame=lines, basicstyle=\tt, tabsize=4, numberstyle=\tt}


\title{HOWTO - Start with the Haarface Detector}
\author{Team BORG}

\begin{document}
\maketitle

\tableofcontents

\section{Introduction}

This document introduces you to the HaarFace detector module.
The HaarFace module allows you to detect faces (not recognize them!).

\textbf{Please note that this document is a work in progress and may be updated in time.}

\section{Installation}

\begin{enumerate}
    \item Install python, opencv and pygame on a linux box:
\begin{lstlisting}
# sudo apt-get update
# sudo apt-get install python python-opencv
\end{lstlisting}
    \item Use Python \lstinline{pip} or install manually if the above does not work.
\end{enumerate}

\section{Basic Training Procedure}

No training should be needed.

\section{Observation Properties}

The following observation properties are available (see example and module for more information):
\begin{enumerate}
    \item \lstinline{pixel_location}: Absolute pixel location of the detection.
    \item \lstinline{relative_location}: Relative location of the detection.
    \item \lstinline{face_area}: Absolute face area.
    \item \lstinline{relative_face_area}: Relative face area, can be used as measurement of the distance of the face.
    \item \lstinline{confidence}: The confidence value, the higher the better.
    \item \lstinline{dimensions}: The dimensions of the camera used (widht height).
\end{enumerate}

\section{Example Usage}

The following files provide you with an example:
\begin{enumerate}
    \item \lstinline{$BORG/brain/src/config/config_haarface_example}
    \item \lstinline{$BORG/brain/src/behavior/examplehaarface/examplehaarface_1.py}
\end{enumerate}

Please refer to the actual ``haarface'' module for more information.

\end{document}


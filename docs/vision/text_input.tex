% [page size, font size, recto-verso]{style of the document}
\documentclass[a4paper, 12pt, oneside]{report}

% Encoding of this file. On Linux, all files are UTF8 encoded.
\usepackage[utf8x]{inputenc}

% To write mathematical formula:
% \begin{align} ... \end{align}, \begin{equation} ... \end{equation}
\usepackage{amsmath, amssymb}

% To insert graphics:
% \includegraphics[options]{path}
\usepackage{graphicx}

% The allowed type of image files.
\DeclareGraphicsExtensions{.pdf, .png}

% To include code file without any layout
% \verbatiminput{path}
\usepackage{verbatim}

% To include PDF files
% \includepdf[pages = z,x-y]{path}
\usepackage{pdfpages}

% To do tables on many pages
% \begin{longtable}{define the columns} ... \end{longtable}
\usepackage{longtable}


\textwidth=16cm
\oddsidemargin=0pt
\evensidemargin=0pt

%% The 3 variables which you must initialize:
% Title of the document,
\newcommand{\titleVariable}{Using text input}
% The author(s),
\newcommand{\authorVariable}{Marko Doornbos}
% The date of the first release.
\newcommand{\firstRelease}{Nov 19, 2012}

\usepackage{hyperref}
% To create all hypertext links for the bibliography, the figures, the bookmarks etc.
% \url{http://www.thewebsite.org}
\hypersetup{
%   backref=true,
%   pagebackref=true,
%   hyperindex=true,
    colorlinks=true,
    breaklinks=true,
    urlcolor= blue,
    linkcolor= blue,
%   bookmarks=true,
    bookmarksopen=true,
    pdftitle={\titleVariable},
    pdfauthor={\authorVariable},
    pdfsubject={Documentation}
}

\title{\titleVariable}
\author{\authorVariable}


\date{\centering First release: \firstRelease \\ Last modification: \today}

\begin{document}
\maketitle

\chapter{General usage}

This explains how to use the $remote\_TextCommand$ module. The module is used to read keyboard text input from a separate window on the executing computer.

\section{Introduction}

The $remote\_TextCommand$ module can be found in the $src/vision$ folder. It reads keyboard text input and writes it to the memory if the enter key is pressed. 

Keep in mind that key presses are only read if the window generated for this purpose is in focus.

\section{How to start}

To have the module running in your behavior, add $remote = remote\_TextCommand$ to the $modules\_settings:$ part and set it to run on the host you want, probably localhost. 

So like: $localhost = remote$ in the $modules:$ section of the config file.

\section{How to read from memory}

Entered commands are stored in memory using the $"text\_command"$ tag. Use that to retrieve them. 

They are stored as tuple of the time the command was stored and the dict containing the actual data. This is the default format for items in memory.

To retrieve the given command from the dictionary, use the $'text\_command'$ key.

\chapter{Other remarks}

None at this time.

\end{document}

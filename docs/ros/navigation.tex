% [page size, font size, recto-verso]{style of the document}
\documentclass[a4paper, 12pt, oneside]{report}

% Encoding of this file. On Linux, all files are UTF8 encoded.
\usepackage[utf8x]{inputenc}

% To write mathematical formula:
% \begin{align} ... \end{align}, \begin{equation} ... \end{equation}
\usepackage{amsmath, amssymb}

% To insert graphics:
% \includegraphics[options]{path}
\usepackage{graphicx}

% The allowed type of image files.
\DeclareGraphicsExtensions{.pdf, .png}

% To include code file without any layout
% \verbatiminput{path}
\usepackage{verbatim}

% To include PDF files
% \includepdf[pages = z,x-y]{path}
\usepackage{pdfpages}

% To do tables on many pages
% \begin{longtable}{define the columns} ... \end{longtable}
\usepackage{longtable}


\textwidth=16cm
\oddsidemargin=0pt
\evensidemargin=0pt

%% The 3 variables which you must initialize:
% Title of the document,
\newcommand{\titleVariable}{Mapping using GMAPPING Documentation}
% The author(s),
\newcommand{\authorVariable}{Amirhosein~Shantia}
% The date of the first release.


\usepackage{hyperref}
% To create all hypertext links for the bibliography, the figures, the bookmarks etc.
% \url{http://www.thewebsite.org}
\hypersetup{
% 	backref=true,
% 	pagebackref=true,
% 	hyperindex=true,
	colorlinks=true,
	breaklinks=true,
	urlcolor= blue,
	linkcolor= blue,
% 	bookmarks=true,
	bookmarksopen=true,
	pdftitle={\titleVariable},
	pdfauthor={\authorVariable},
	pdfsubject={Documentation}
}

\title{\titleVariable}
\author{\authorVariable}


\date{\centering Last modification: \today}

\begin{document}
\maketitle

\begin{abstract}
\end{abstract}

% To create a table of contents at the beginning of the documentation.
\tableofcontents

\section{Introduction}

This documentation helps you to understand how to use Gmapping method with the BORG architecure.

\section{Pre Requisites}
The pre requisites to run gmapping is as follow:

\begin{enumerate}
 \item ROS Fuerte installation
 \item sicktoolbox and sicktoolbox\_wrapper stack
 \item gmapping stack
 \item borg\_pioneer stack 
 \item Borg architecture with manual or automatic moving behavior
\end{enumerate}

It is better for you to know your way around ROS. So I suggest that you take a look at ROS tutorial at 
\url{www.ros.org}. Especially the commands to find,install, and run packages. If the laptop that you are working with is
already set up by BORG members, you don't have to install anything.

\section{Laser Drivers and Stack}
The most important thing is to have the laser running. So make sure that the ros-fuerte-laser-drivers package is installed. If not,
use ubuntu package manager to install it. That will install the sicktoolbox stack. You also need the wrapper to actually access the laser.
Use \verb|rosmake sicktoolbox_wrapper| in your ROS workspace to make the package.\\


You need to note that the laser is connected through a serial-to-usb connector. This means it is available in one of the /dev/ttyUSBX, where X is 
a number between 0-9. Since our pioneer is also using serial cables, you need to check which number is the correct one. Usually disconnect and 
connect again, then the highest number is for the laser.\\

After you have found the port you need to tell the wrapper, and dont worry about the baud rate. It will find it automatically.
In order to set parameters you have three choices. 

\begin{enumerate}
 \item Setting parameters before running: \\
 \verb|rosparam set sicklms/port /dev/ttyUSBX|
 \item Running and setting parameters in one line:\\
 \verb|rosrun sicktoolbox_wrapper sicklms _port:=/dev/XXX |
 \item XML roslaunch file. I will talk about this later
\end{enumerate}


More information about sicktoolbox\_wrapper can be found in \url{http://www.ros.org/wiki/sicktoolbox_wrapper}

Now, in order to run the laser node, you have to use the \verb|rosrun| command as stated before. 	

\section{GMAPPING}
Gmapping is one of the strongest laser based SLAM method developed by MIT. It is installed automatically with ros fuerte desktop full package.
In order to run it simply usepackage
\begin{verbatim}
 rosrun gmapping slam_gmapping
\end{verbatim}

For more information check \url{http://www.ros.org/wiki/gmapping}

\section{Borg pioneer}
This ros stack is available in the repo. Just go to \verb|robotica/ros/borg_pioneer| and use \verb|rosmake borg_pioneer| to make the package.
There is a transform cpp file that will be built. Use rosrun to run it as follow.
\begin{verbatim}
 rosrun borg_pioneer transform
\end{verbatim}

Now, everything is ready. Just run a behavior with manual or automatic moving. As soon as the robot moves, you should see information on the gmapping
terminal. If there are warning about \verb|odom| topic, then something is wrong. Ask an expert!

In order to see what is being mapped you should run \verb|rviz|, ROS vizualization program. Just open a new terminal and type
\begin{verbatim}
 rosrun rviz rviz
\end{verbatim}

You should see the location of the robot, the current view of the laser. As soon as you move, you see the map growing, and the robot location on the 
map too.

\section{Map server}
After you mapped the environment, you probably want to save it! this is done by using the \textit{map\_server} stack.
You type 

\begin{verbatim}
rosrun map_server map_saver Name_of_the_map_to_be_saved.       
\end{verbatim}

Then, the program makes two files, one picture and one yaml. These can be loaded later for localization using other stacks.

\section{Adaptive Monte Carlo Localization - AMCL}

Now that a map is made, we can use AMCL for offline localization. The amcl package is part of the ROS navigation stack.
This method requires a pose estimate when the module is loaded. \textbf{WARNING}, if the pose estimation is wrong, your whole localization will be rubbish.
This pose estimation can be done either using rviz, or you can enter a starting pose.\\
In rviz, on top left, there is a button names \textit{2D Pose Estimate}. Click it, then click(and keep holding) the approximate location of robot, then move the mouse
to select the correct angle. You will see a green arrow, and as soon as you release the mouse button, the pose estimate will be sent to amcl.\\
You can also enter a pose estimate in the config file as follow:

\begin{verbatim}
#IMPORTANT MAP LOCATIONS:
#				  X     Y   Angle
#	DOOR A: pioneer_pose_0 = -1500 -500 90
#
#	DOOR B: pioneer_pose_0 = -9000 13500 0
#PIONEER STARTING POSE
pioneer_pose_0 = 0 0 0

\end{verbatim}

As can be seen, we usually keep important locations in the config file as comment for fast switching of the starting pose.


\section{Good Practices for Mapping}
It is usually a good idea to select a corner of a room/arena as the coordinate (0 , 0, 0). It is better than this corner is close to one of the
doors. Then use a meter or rviz to read to estimate the distance of the corner to the middle of the door. Since your robot will usually stand 1 meter behind the door, 
that location is a proper starting pose. Calculate the rest of the entries/exits using the same method.

\section{Path Planning}
At this moment we use our own path planning method. During mapping, you can save locations and give them names... To be continued.

\end{document}

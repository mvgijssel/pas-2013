\documentclass[a4paper, 10pt, oneside]{article}
\usepackage[utf8x]{inputenc}
\usepackage{amsmath, amssymb}
\usepackage{graphicx}
\DeclareGraphicsExtensions{.pdf, .png}
\usepackage{verbatim}
\usepackage{hyperref}

\usepackage{listings}

\usepackage{courier}


%TODO: dit moet mooier kunnen.....
\newcommand{\cod}[1]{\emph{#1}}

\title{The memory}
\begin{document}
\maketitle


%set the font size of the code examples:
\lstset{ 
basicstyle=\footnotesize,  
}


\section{Introduction}

This document is about the memory in the Brain of the architecture. Its not about the shared memory etc.


\section{The memory class}

The memory class is a singleton, which means that there will only be one memory, no matter how much memory objects you create.
All behaviors have the memory already accessible via self.m , but in all other classes you can just say:

\begin{lstlisting}
self.m = memory.Memory()
\end{lstlisting}

This will get you the current active memory.

\subsection{Adding things to memory}

To add something to the memory, just use the 

\begin{lstlisting}
self.m.add_item(name, time, properties)
\end{lstlisting}

The time you can get with time.Time(). The properties are in a dictionary, for example:

\begin{lstlisting}
properties = {'color': 'green','length': 20}
\end{lstlisting}

\subsection{Reading things from memory}

The memory remembers all the additions done since it was started. Usually, you only want to get the most recent addition of some object.
You do that as follows:

\begin{lstlisting}
m.get_last_observation(name)
\end{lstlisting}

Where name is the name of the object.

You can also get a list of all the observations:

\begin{lstlisting}
m.get_observations(name)
\end{lstlisting}

This will get you a list of all the observations, ordered by time.

Two other useful functions check for an object in combination with its properties: is\_now and was\_ever.

For example

\begin{lstlisting}
m.is_now('cup', {'color':'red'})
\end{lstlisting}

checks if we are seeing a red cup.

\begin{lstlisting}
m.was_ever('location', {'room':'kitchen'})
\end{lstlisting}

checks if we have ever been in the kitchen. These functions can be used for example in the pre and postconditions of the behaviors.




\section{Vision modules}

Vision modules can register the objects they can possibly recognize via:

\begin{lstlisting}
m.add_recognizable_objects(modulename,objects)
\end{lstlisting}

Where objects is just a list of the object names the module can detect. The behaviors can then just check for what kind of objects
we can currently recognize by using:

\begin{lstlisting}
m.is_object_recognizable(object)
\end{lstlisting}

When a module is stopped, it should call:

\begin{lstlisting}
m.clean_recognizable_objects(modulename)
\end{lstlisting}

This removes the objects that module can detect from the memory (note that this is completely separate from observations in the memory, those are never removed).
When multiple modules can recognize the same object, that object remains in the list until all the modules have cleared their objects.

See the ``example'' module for a working example.

\section{Logging}

It is possible to let the memory log everything that it stores, and it is also possible to import a log back to restore an earlier state of memory.
This is done with the load\_log() function. 

This still needs to be made configurable in the config. When that is done, more information will be added here....



\end{document}

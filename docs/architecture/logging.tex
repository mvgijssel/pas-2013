\documentclass[a4paper, 10pt, oneside]{article}
\usepackage[utf8x]{inputenc}
\usepackage{amsmath, amssymb}
\usepackage{graphicx}
\DeclareGraphicsExtensions{.pdf, .png}
\usepackage{verbatim}
\usepackage{hyperref}

\usepackage{listings}


\title{Logging facilities}
\author{Egbert van der Wal}
\begin{document}
\maketitle

\section{Usage}
In all classes I altered, I added the logger initializer in the constructor. This means that from anywhere in the class, you can do any of:

\begin{lstlisting}
self.logger.debug("My debug message")
self.logger.info("My information message")
self.logger.warning("My warning")
self.logger.error("My error message")
self.logger.critical("My critical error message")
\end{lstlisting}

There were a couple of files with no class (are they used?), but with just functions. To those files, I added a global variable named logger, so from within those functions you can do:

\begin{lstlisting}
logger.debug("My debug message")
\end{lstlisting}

and so on.


If you use this, please try to use the appriopriate log-level:
\begin{itemize}
\item Debug is for extreme verbosity only, you can, for example, dump bits of the content of the memory with this.
\item Info is for more meaningful messages, still not required for normal functioning.
\item Warning is when you detect things go some way they shouldn't but the code is trying to work around it.
\item Error is for messages when the code encounters a problem and cannot work around it, and thus aborts, while the program can remain running, possibly with decreased functionality
\item Critical is for problems that require immediate termination of the program.
\end{itemize}


\section{DIY}
Because it's quicker and more comprehensible to give an example, I'll just post a small class that does everything it needs.

\begin{lstlisting}
# -----------------------------------------
import logging
import util.nullhandler

logging.getLogger('Borg.Brain.LogExample'). \
    addHandler(util.nullhandler.NullHander())

class LogExample(object):
    def __init__(self):
        self.logger = logging.getLogger('Borg.Brain.LogExample')

    def exampleFunction(self):
        self.logger.info("Here is my warning message")
# -----------------------------------------
\end{lstlisting}


\section{Enabling Output}
You see the name of the logger in the example above as: Borg.Brain.LogExample.
The name is hierarchical where the parts are separated by dots. Using this
hierarchy, you can enable or disable logging for smaller or larger parts at
once, or change the logging level for smaller or larger parts at once.\\

When you want to get output from a specific module, you use this code IN YOUR
MAIN FUNCTION. Do not touch the line adding the NullHandler from the example
above (the 3rd line with code). Use the following code somewhere before you
expect logging output. This will enable logging for the entire brain. \\

\begin{lstlisting}
logging.getLogger('Borg.Brain'). \
    addHandler(util.loggingextra.ScreenOutput())
logging.getLogger('Borg.Brain').setLevel(logging.DEBUG)
\end{lstlisting}

Don't forget to import logging and util.loggingextra!

\section{Logging in base behaviors}
Because the same code is used for all base behaviors, they all use the same
logger name, Borg.Brian.Behavior. You can change this if you want and if it's
necessary.

\section{Command Line Options}
The brain.py accept logging options. You can use --log-name to specify what to
log, --log-level to specify the log level (DEBUG, INFO, WARNING, ERROR,
CRITICAL), --log-format to specify the logging format (although you shouldn't
usually need this) and --log-file to specify a file to log to instead of the
standard output. Here are a couple of examples: \\

\begin{lstlisting}
./start.sh --log-name=Borg.Brain.Behavior --log-level=DEBUG 
\end{lstlisting}
to enable output of all messages of the Behaviors.\\
\\
Another option is to specify a target file:\\
\begin{lstlisting}
./start.sh --log-name="Borg.Brain.Behavior" --log-level=DEBUG \
    --log-file=brain.log
\end{lstlisting}

\section{Online manual}
See: http://docs.python.org/library/logging.html for more information on the logging facilities.
\end{document}


% [page size, font size, recto-verso]{style of the document}
\documentclass[a4paper, 12pt, oneside]{report}

% Encoding of this file. On Linux, all files are UTF8 encoded.
\usepackage[utf8x]{inputenc}

% To write mathematical formula:
% \begin{align} ... \end{align}, \begin{equation} ... \end{equation}
\usepackage{amsmath, amssymb}

% To insert graphics:
% \includegraphics[options]{path}
\usepackage{graphicx}

% The allowed type of image files.
\DeclareGraphicsExtensions{.pdf, .png}

% To include code file without any layout
% \verbatiminput{path}
\usepackage{verbatim}

% To include PDF files
% \includepdf[pages = z,x-y]{path}
\usepackage{pdfpages}

% To do tables on many pages
% \begin{longtable}{define the columns} ... \end{longtable}
\usepackage{longtable}


\textwidth=16cm
\oddsidemargin=0pt
\evensidemargin=0pt

%% The 3 variables which you must initialize:
% Title of the document,
\newcommand{\titleVariable}{Pioneer Controller Documentation}
% The author(s),
\newcommand{\authorVariable}{Herke~van~Hoof}
% The date of the first release.
\newcommand{\firstRelease}{March 18, 2011}

\usepackage{hyperref}
% To create all hypertext links for the bibliography, the figures, the bookmarks etc.
% \url{http://www.thewebsite.org}
\hypersetup{
% 	backref=true,
% 	pagebackref=true,
% 	hyperindex=true,
	colorlinks=true,
	breaklinks=true,
	urlcolor= blue,
	linkcolor= blue,
% 	bookmarks=true,
	bookmarksopen=true,
	pdftitle={\titleVariable},
	pdfauthor={\authorVariable},
	pdfsubject={Documentation}
}

\title{\titleVariable}
\author{\authorVariable}


\date{\centering First release: \firstRelease \\ Last modification: \today}

\begin{document}
\maketitle

\begin{abstract}
\end{abstract}

% To create a table of contents at the beginning of the documentation.
\tableofcontents

\section{Introduction}
This documentation aims to describe the structure and buildup of the pioneer
controller. This is a program intended to run on a pioneer robot. It receives
and interprets commands send through a Tcp/Ip connection and subsequently
executes the command. The pioneer controller program was written to allow
control of the pioneer by programs running at remote machines, regarding of the
language in which they are written. Specifically, it was created to allow the
python `robot brain' developed by the `BORG' team of the University of Groningen
for participation in the international Robocup@home competition to control the
robot.

\subsection{History}
An earlier program called the `CursorServer' provided a similar function,
receiving, interpreting, and executing commands send over a Tcp/Ip connection.
However, this program aims to provide better control. Also, I've tried to split
up code as much as possible, allowing for code reuse and flexible improvement
(allowing one to for instance improve the Tcp/Ip server without having to
modify, or to have any knowledge of, the subsequent parsing and execution of the
commands).

\subsection{Quality}
The pioneer controller has been significantly improved since the initial
release. It will currently try to keep binding to the proper port instead of
giving up, it will try to reconnect to the Pioneer Controller instead of
crashing when the connection is lost, it can handle more than one connection at
the same time and the error handling has been improved. Still, more improvements
are always possible. \\

The aim of creating the pioneer controller was to provide a working example of
all modules it consists of (a Tcp/Ip server, a parser, and the actual motion
controller), not to create a perfect version of any of these. Therefore, there
is probably plenty of space for improvement in each of these modules. The
pioneer controller comes with a set of unit tests, please keep these up to date
and add tests for new functionality that is added.

\subsection{This document}
Like the pioneer controller, this document is likely to be imperfect. If you
spot any errors, please correct them or add additional information as necessary
(and add your name in the author section if you think your contribution was
significant). Also if new functionality is added, please describe it in this
document. The person after you to use this document will be grateful!

\section{Preparation}
Before being able to run the pioneer controller, the code should be copied to a
pioneer. Also, any necessary libraries and a compiler need to be available.
Then, the code can be compiled. Then, the program is ready to be executed. This
chapter will explain all of these steps.

\subsection{Copying the code}
Code can be copied to the pioneer using scp:

\begin{verbatim}
scp -r [folder] [user]@[host]:[location]
\end{verbatim}

In this command, [folder] is the folder you want to copy (for instance
`pioneercontroller'), [user] the username on the remote system (for instance
`root'), [host] is the hostname of the pioneer (for instance `ai178157' for the
robot labeled pippin), and [location] is the location on the host machine where
the folder has to be copied to. Leave this blank to copy the source to the home
directory of the [user]. The example command obtained is:

\begin{verbatim}
scp -r pioneercontroller root@ai178157:
\end{verbatim}

Scp will prompt you for the password of [user], and subsequently copy the code
as requested. If you forget the colon (`:'), the folder contents will be copied
to a folder called `root@ai178157' in your current local directory, which is not
what you want.

\subsection{Necessary libraries and compiler}
Please make sure a compiler is installed. I used \verb=g++= (version 4.1.3),
other version might or might not work. Also, please make sure the Aria library
(by MobileRobots inc.) is installed. I used version 2.7.2. For unit testing, the
`cppunit' library is needed (\url{http://cppunit.sourceforge.net/doc/1.8.0/}).

\subsection{Compiling the code}
For compiling the code, a `Makefile' is available. The Makefile will more or
less intelligently try to find the proper location of the libraries and include
files and should succeed at least on the Pioneer and on the lab computers. If
you experience problems, feel free to fix them, however, try to avoid hardcoding
more library locations in the Makefile. Instead, use a better method to find the
proper locations automatically.

By the way, `make' or `make all' compile both the control program and the tests.
`make motionController' compiles just the control program, `make
motionControllerTest' compiles just the tests. `make clean' removes all object
files and the linked binaries.

\section{Testing and running the program}
Now the program and the tests are compiled they can be run. If the cppunit
library is installed in an alternative location, this should be specified when
running the tests: 

\begin{verbatim}
LD_LIBRARY_PATH=/root/cppunit/lib/ ./motionControllerTest
\end{verbatim}

Of course, all tests should succeed. In this case, running the tests should echo
`OK(2 tests)'.

Running the pioneer controller is easy: `./motionController [portnumber]'. When
the program is running, it will first wait for an incoming Tcp/Ip connection on
the specified port, and then try to read from it every fraction of a second.
Received commands are executed. If no command is received for 0.5 seconds, speed
of the robot is reset to 0 to prevent accidents. So the control program sending
the commands should in normal situations send one command every 0.5 seconds or
more often.

\section{Structure of the source code}
As mentioned in the introduction, the functionality is divided over three
classes: the Tcp/Ip server, the parser, and the motion controller. A main
function used the Tcp/Ip server to receive the next command, which is a string.
This is done in non-blocking manner, such that if there is no command the server
returns an empty string and execution of the program can continue (more clearly:
the server does not wait for a command to arrive).\\

The TCP/IP server accepts several simultaneous connections. Each new connection
will start in a released state. This means that it cannot currently control the
robot. It first has to send the command 'enslave' to do so. This will instruct
the RobotController to connect to the robot. Disconnecting from the robot
without disconnecting from the server can be done by giving the 'release'
command. When no clients are in an 'enslaved' state, the RobotController will
disconnect from the robot. The only command that can be sent in the released
state is 'status'. This will return some information about the current
situation: whether there is an active connection to the robot, the current
odometry and battery levels and the number of connections. \\

Each connection is handled in a separate thread currently. A better approach
might be to use epoll to handle multiple connection simultaneously, but this
requires some more work, and as there usually are at most 2 connections to the
Pioneer at the same, this is not really a problem. \\

The command string is parsed by the parser. Currently, the parser returns a pair
with as its first member a bool indicating whether the string was successfully
parsed. If so, the second member is a pair containing the speeds to be set for
the left and right wheel engines respectively. \\

This command is then executed by the motion controller, interupting any command
received earlier. However, if no parsable command was received for more than 0.5
seconds, the control program sending the commands is assumed to be dead. In this
case, for security the speed of the pioneer is set to 0 for both engines. \\

\subsection{Unit tests}
At this moment, unit tests are provided for the CommandParser::parseCommand and
TcpIpServer::receiveMsg functions. The only other function is setSpeeds, which
can't really be tested as it only provides side effects on the pioneer. All
tests are implemented in the `tester.cpp' file.

\end{document}

%l [page size, font size, recto-verso]{style of the document}
\documentclass[a4paper, 12pt, oneside]{article}

% Encoding of this file. On Linux, all files are UTF8 encoded.
\usepackage[utf8x]{inputenc}

%Used to make nice looking listings:
\usepackage{listings}

% To write mathematical formula:
% \begin{align} ... \end{align}, \begin{equation} ... \end{equation}
\usepackage{amsmath, amssymb}

% To insert graphics:
% \includegraphics[options]{path}
\usepackage{graphicx}

% The allowed type of image files.
\DeclareGraphicsExtensions{.pdf, .png}

% To include code file without any layout
% \verbatiminput{path}
\usepackage{verbatim}

% To include PDF files
% \includepdf[pages = z,x-y]{path}
\usepackage{pdfpages}

% To do tables on many pages
% \begin{longtable}{define the columns} ... \end{longtable}
\usepackage{longtable}


\textwidth=16cm
\oddsidemargin=0pt
\evensidemargin=0pt

%% The 3 variables which you must initialize:
% Title of the document,
\newcommand{\titleVariable}{PAS1: Soccer Regulations}
% The author(s),
\newcommand{\authorVariable}{}
% The date of the first release.
\newcommand{\firstRelease}{2013-03-12}

\usepackage{hyperref}
% To create all hypertext links for the bibliography, the figures, the bookmarks etc.
% \url{http://www.thewebsite.org}
\hypersetup{
% 	backref=true,
% 	pagebackref=true,
% 	hyperindex=true,
	colorlinks=true,
	breaklinks=true,
	urlcolor= blue,
	linkcolor= blue,
% 	bookmarks=true,
	bookmarksopen=true,
	pdftitle={\titleVariable},
	pdfauthor={\authorVariable},
	pdfsubject={Documentation}
}

\title{\titleVariable}
\author{\authorVariable}


\date{\centering First release: \firstRelease \\ Last modification: \today}

\begin{document}
\maketitle

\section{The Game}

The purpose of the game is to have each team put the soccer ball into its opponent's goal.

\subsection{A Single Match}

Every single match has the following schedule (in minutes):

\begin{enumerate}
    \item[00:00 - 05:00:] Setup time for both teams.
    \item[05:00 - 10:00:] First half. 
        The team that is mentioned first on the schedule has goal \#1.
    \item[10:00 - 14:00:] Setup time for both teams.
    \item[14:00 - 19:00:] Second half. 
        The team that is mentioned second on the schedule has goal \#1
    \item[19:00 - 20:00:] Cleanup time (make sure to leave the green area within 1 minute).
\end{enumerate}

\subsection{Type of Matches}

\begin{enumerate}
    \item Type 1:
    \begin{enumerate}
        \item Team 1: One attacker.
        \item Team 2: One defender.
    \end{enumerate}
    \item Type 2:
    \begin{enumerate}
        \item Team 1: One attacker and one defender.
        \item Team 2: One attacker and one defender.
    \end{enumerate}
\end{enumerate}

\subsection{Weekly Matches}

\subsubsection{Quarterfinals}

In first round there will be the same amount of matches as there are teams (7 in 2013).
Each match has 2 halves, each team will thus play twice, every time with a \emph{different} opponent.
If you win a match, you will get 3 points. If there is a tie, each team will get 1 point.

\subsubsection{Semifinals}

The best 4 teams will continue to the semifinals.
Which will result in 2 matches.

\subsubsection{Finals}

The winners of the semifinals will continue to the finals.
This will result result in 1 match in which the winner becomes champion!

\section{Arena}

\subsection{Dimensions} 
\label{sec:dimensions}

\begin{enumerate}
    \item[\textbf{arena length:}] ?cm
    \item[\textbf{arena width:}] ?cm
    \item[\textbf{goal width:}] 80cm (in-between the poles)
    \item[\textbf{dead zone:}] area within 30cm from the wall except right in front of the goals.
\end{enumerate}

\subsection{Locations}

South is towards the wall with the doors.

\begin{enumerate}
    \item[\textbf{left wall:}] on the west side of the lab
    \item[\textbf{right wall:}] on the east side of the lab
    \item[\textbf{goal \#1:}] in the middle of the south wall
    \item[\textbf{goal \#2:}] in the middle of the north wall
    \item[\textbf{left marker:}] in the middle of the left wall
    \item[\textbf{right marker:}] in the middle of the right wall
    \item[\textbf{ball kickoff:}] at the very center of the arena
\end{enumerate}

\subsection{Colors}

\begin{enumerate}
    \item[\textbf{floor:}] dark green
    \item[\textbf{ball:}] shiny red
    \item[\textbf{goal \#1:}] yellow
    \item[\textbf{goal \#2:}] blue
    \item[\textbf{left marker:}] blue/yellow banner (top: yellow, bottom: blue)
    \item[\textbf{right marker:}] blue/yellow banner (top: blue, bottom: yellow)
    \item[\textbf{robots:}] grey/white
\end{enumerate}

\section{Rules}

\subsection{Humans in the Arena}

Humans are not allowed in the arena unless:
\begin{enumerate}
    \item There is not a match going on (this includes setup time).
    \item It is the referee.
    \item It is a team member doing any of the following:
    \begin{enumerate}
        \item Guiding the cables connected to the robot (touching the robot results in a robot restart).
        \item Picking up the robot or the ball for the kick-off position or after a time-out penalty.
    \end{enumerate}
\end{enumerate}

Humans should not be wearing shoes while being inside the arena.
Robots are not allowed to wear any shoes at any time. 

\subsection{Time-out Penalty}

A timeout penalty is given to any robot that does not comply to the rules stated in this document.
During the time-out penalty, the robot is removed from the arena for the specified amount of time.
The time-out penalty of a robot can be extended if the corresponding team decides to do so.

Whenever a time-out penalty passes, the robot will be placed to the closest edge of the dead zone from where the penalty was issued.
This with the exception of the defender which will be placed inside the arena from its initial position.

\subsection{(Re)Starting the Robot}

The robot can be (re)started whenever:

\begin{enumerate}
    \item A kick-off is issued.
    \item After a timeout penalty passed.
    \item When the teams decides to do so at the cost of a time-out penalty of 30 seconds\footnote{This can be useful in case your robot crashes or you want to tweak a few incorrect parameters.}.
\end{enumerate}

Restarting the robot should only require pushing a single button (no in-game tweaking is allowed inside the arena).

\subsection{Abnormal Robot Behavior}

Whenever a robot behaves abnormal, a time-out penalty of 30 seconds is issued for the corresponding robot. 
Abnormal robot behavior includes:
\begin{enumerate}
    \item Lying on the floor for more than 10 seconds.
    \item Standing still (without moving anything) for more than 10 seconds.
    \item Moving the limbs in a way so it will damage itself or anything else.
\end{enumerate}

\subsection{Manual Control}

The robot can only be controlled manually under the following conditions:
\begin{enumerate}
    \item While the robot is outside the arena (e.d. while waiting for the timeout penalty).
    \item During the setup time of each half.
    \item Right before a restart of a robot is issued (by pushing a single button).
\end{enumerate}

Manual control includes touching the robot or any computers that controls the robot.

\subsection{Collisions}

\begin{enumerate}
    \item If a robot collides into a wall, a timeout penalty of 15 seconds is issued for that robot.
    \item If two or more robots collide into each other, then the one causing the collision will get a timeout penalty of 15 seconds.
        However if the other robot falls because of the collision, then a kickoff is issued in favour of the team that did not cause the collision.
        The kickoff in this case will be at the same spot of the collision.
    \item If two or more robots collide into each, and they both caused the collision, a new kickoff is issued.
        The team doing the actual kickoff is picked using a coin-flip.
    %\item A referee may decide to pick up a robot and issue a timeout penalty for that robot whenever he or she thinks that the robot is \emph{about to} collide into an object or another robot.
\end{enumerate}

\subsection{Ball Location}

\begin{enumerate}
    \item If a robot kicks the ball outside the arena, a new kickoff is issued in favour of the opponents.
    \item If the ball does not move for more than 10 seconds inside the dead zone (see section \ref{sec:dimensions}), a new kickoff is issued in favour of the team that did not cause it.
\end{enumerate}

\subsection{Defender}

\begin{enumerate}
    \item Initial position (after restart): the defender should be placed within 80 cm from the goal (from either pole).
    %\item During the whole game, every defender should remain in their own half. 
        %If they leave their half, they get a timeout penalty of 30 seconds. 
\end{enumerate}

\subsection{Attacker}

\begin{enumerate}
    \item During kickoff, all attackers should be placed within their own half.
    \item The attacker doing the kickoff can be placed right in front of the ball.
    \item The attacker \emph{not} doing the kickoff should initially be placed inside its own half with a distance of 50cm from the ball.
\end{enumerate}

\subsection{Disqualification}

Any referee may decide to disqualify a robot or its team for any half or the whole competition for, but not limited to, demonstrating the following behavior:
\begin{enumerate}
    \item Jumping (humans are allowed to jump).
    \item Kicking, slapping, strangling and spitting of/to other robots/humans on purpose.
\end{enumerate}

\end{document}
